%%%%%%%%%%%%%%%%%%%%%%%%%%%%%%%%%%%%%%%%%
% FRI Data Science_report LaTeX Template
% Version 1.0 (28/1/2020)
% 
% Jure Demšar (jure.demsar@fri.uni-lj.si)
%
% Based on MicromouseSymp article template by:
% Mathias Legrand (legrand.mathias@gmail.com) 
% With extensive modifications by:
% Antonio Valente (antonio.luis.valente@gmail.com)
%
% License:
% CC BY-NC-SA 3.0 (http://creativecommons.org/licenses/by-nc-sa/3.0/)
%
%%%%%%%%%%%%%%%%%%%%%%%%%%%%%%%%%%%%%%%%%


%----------------------------------------------------------------------------------------
%	PACKAGES AND OTHER DOCUMENT CONFIGURATIONS
%----------------------------------------------------------------------------------------
\documentclass[fleqn,moreauthors,10pt]{ds_report}
\usepackage[english]{babel}

\graphicspath{{fig/}}




%----------------------------------------------------------------------------------------
%	ARTICLE INFORMATION
%----------------------------------------------------------------------------------------

% Header
\JournalInfo{FRI Natural language processing course 2023}

% Interim or final report
\Archive{Project report} 
%\Archive{Final report} 

% Article title
\PaperTitle{An Automatic Movie Summary Generator} 

% Authors (student competitors) and their info
\Authors{Luka Pavićević, Andrija Stanišić, Stefanela Stevanović}

% Advisors
\affiliation{\textit{Advisor: Slavko Žitnik}}

% Keywords
\Keywords{}
\newcommand{\keywordname}{Keywords}


%----------------------------------------------------------------------------------------
%	ABSTRACT
%----------------------------------------------------------------------------------------

\Abstract{
}

%----------------------------------------------------------------------------------------

\begin{document}

% Makes all text pages the same height
\flushbottom 

% Print the title and abstract box
\maketitle 

% Removes page numbering from the first page
\thispagestyle{empty} 

%----------------------------------------------------------------------------------------
%	ARTICLE CONTENTS
%----------------------------------------------------------------------------------------

\section*{Introduction}

\section*{Related works}

One of the related works in text summarization is the work by Aleš Žagar and Marko  Robnik-Šikonja \cite{zagar2022}, who explored the use of various summarization approaches, including neural models, to produce short summaries from larger texts. In their work, they addressed the problem of selecting the most appropriate summarization model for a given text, and proposed a solution that uses a neural metamodel to automate the selection process based on the input document's representation. This work presents an innovative approach to text summarization, particularly in addressing the issue of model selection, which is important in ensuring the quality of generated summaries.

Another related work \cite{subramanian2017automatic} describes a system for automatically generating movie descriptions using subtitles and other metadata. The authors use a combination of NLP techniques such as named entity recognition, sentiment analysis, and topic modeling to extract relevant information from the subtitles and generate concise descriptions.

In addition, another work \cite{huang2020movie} present a movie summarization system that uses subtitles and user ratings to generate short descriptions of movies. The authors use a combination of sentence clustering, topic modeling, and sentiment analysis to extract key information from the subtitles.

Furthermore, a recent work \cite{chen2019generating} presents a neural language model for generating movie summaries from subtitles. The authors use a combination of convolutional and recurrent neural networks with a latent variable model to generate concise and coherent summaries.

Lastly, a recent work \cite{jhanwar2020movie} proposes a system for generating movie synopses using subtitles and plot summaries. The authors use a combination of named entity recognition, sentiment analysis, and topic modeling to extract relevant information and generate summaries that capture the essence of the movie.




%----------------------------------------------------------------------------------------
%	REFERENCE LIST
%----------------------------------------------------------------------------------------
\bibliographystyle{unsrt}
\bibliography{report}
 

\end{document}